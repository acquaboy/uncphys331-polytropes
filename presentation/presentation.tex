\documentclass{beamer}

\usepackage{amsmath}

\usetheme{Dresden}
\usecolortheme{default}
\usefonttheme{serif}
\usefonttheme[onlylarge]{structuresmallcapsserif}
%\usefonttheme[onlysmall]{structurebold}

\title[Polytropes] % OPTIONAL: only for long titles
{Polytropic Models of White Dwarfs}
\subtitle{UNC PHYS 331 Project}
\author[Conn, Hurley] % OPTIONAL: for multiple authors
{Erin Conn \and Matthew Hurley}
\subject{Polytropes}

\AtBeginSection[]
{
        \begin{frame}<beamer>
            \frametitle{Table of Contents}
            \tableofcontents[currentsection]
        \end{frame}
}


\AtBeginSubsection[]
{
    \begin{frame}<beamer>
        \frametitle{Table of Contents}
        \tableofcontents[currentsection,currentsubsection]
    \end{frame}
}

\begin{document}

    \frame{\titlepage}

    \section{Theory}

        \subsection{Polytropes}

        \begin{frame}
            \frametitle{What are polytropes?}

            Solutions to...

            \textbf{The Lane-Emden Equation}

            \[
                \frac{1}{\xi^2}\frac{d}{d\xi}\left(\xi^2\frac{d\theta}{d\xi}\right)=-\theta^n(\xi)
            \]

            A dimensionless, 2nd order nonlinear differential equation relating the
            pressure of a spherically-symmetric gas distribution to the radius.

        \end{frame}

        \begin{frame}
            \frametitle{Why polytropes?}

            \begin{itemize}
                \item Provide simplified stellar models - simple pressure/density relation
                    \pause
                \item Easier to solve than full equations of stellar structure
                    \pause
                \item Require less computational effort - some analytic solutions even exist!
            \end{itemize}

        \end{frame}

        \begin{frame}
            \frametitle{Definitions}

            \begin{definition}
                \alert{Polytropic process} - Thermodynamic process that obeys the relation
                
                \[PV^n=C\]
            \end{definition} 
                    \pause
            \begin{definition}
                \alert{Polytropic index} - Constant that relates pressure of a polytropic fluid to its volume (density). It may be any real number.
            \end{definition}
            \pause
            \begin{definition}
                \alert{Poisson's equation} Relates a force density function to a potential field
                    \[\nabla^2\Phi=f\]
            \end{definition}

        \end{frame}

        \begin{frame}
            \frametitle{Derivation 1: Poisson Equation}
            % Skip this and the next 2 slides if pressed for time
            % Look up how to add skip buttons
            Can be derived multiple ways. From laws of mass conservation and
            hydrostatic equilibrium:
            \pause
            \[ \begin{array}{l l}
                dM(r) &= 4\pi r^2\rho(r)dr \rightarrow \frac{dM(r)}{dr}=4\pi r^2\rho(r) \\
                \pause
                \frac{dP(r)}{dr} &= -\frac{\rho(r)GM(r)}{r^2}
                \end{array}
            \]
            \pause
            These equations are related by multiplying the hydrostatic equation by
            \(r^2/\rho\) and differentiating:
            \pause
            \[\frac{d}{dr}\left(\frac{r^2}{\rho(r)}\frac{dP(r)}{dr}\right)=-G\frac{dM(r)}{dr}\]
            \pause
            Yielding Poisson's equation for gravity:
            \pause
            \[ \frac{1}{r^2} \frac{d}{dr} \left( \frac{r^2}{\rho(r)} \frac{dP(r)}{dr} \right) =-4 \pi G \rho(r) \]
        \end{frame}
        \begin{frame}
            \frametitle{Derivation 2: Working towards a dimensionless form}

            Define a polytropic state equation:
            \pause
            \[ P=K\rho^{\frac{n+1}{n}} \]
            \pause
            Make it dimensionless:
            \[\theta^n\equiv\frac{\rho}{\rho_c} \]
            \pause
            \[ P(r)=K\rho_c^{\frac{n+1}{n}}\theta^{n+1}(r)=P_c\theta^{n+1}(r) \]
            \pause
            Substitute into Poisson and simplify:
            \pause
            \[\frac{(n+1)P_c}{4\pi G\rho_c^2}\frac{1}{r^2}\frac{d}{dr}\left(r^2\frac{d\theta(r)}{dr}\right)=-\theta^n(r) \]

        \end{frame}

        \begin{frame}
            \frametitle{Derivation 3: More simplification}

            Define a new variable:
            \pause
            \[\alpha^2 \equiv \frac{(n+1)P_c}{4\pi G\rho_c^2}\]
            \pause
            Use it to define a dimensionless radius:
            \pause
            \[\xi \equiv \frac{r}{\alpha}\]
            \pause
            Substitute into the simplified Poisson:
            \pause
            \[\frac{1}{\xi^2}\frac{d}{d\xi}\left(\xi^2\frac{d\theta(\xi)}{d\xi}\right)=-\theta^n(\xi)\]

        \end{frame}

        \subsection{White Dwarfs}

        \begin{frame}
            \frametitle{Placeholder}

            Something about degenerate matter and how polytropic models suit it?

        \end{frame}

        \begin{frame}
            \frametitle{Placeholder}

            Relativistic vs. Non-Relativistic?

        \end{frame}

    \section{Methods}

        \begin{frame}
            \frametitle{Alternate form of the Lane-Emden Equation}

            \[\frac{d^2\theta}{d\xi^2}=-\frac{2}{\xi}\frac{d\theta}{d\xi}-\theta^n(\xi)\]

        \end{frame}

        \begin{frame}
            \frametitle{Translating to a system of 1st order equations}

            \[
                \left\{ \begin{array}{l l}
                \phi &= \frac{d\theta}{d\xi} \\
                \frac{d\phi}{d\xi} &= -\frac{2}{\xi}\phi - \theta^n
                \end{array} \right.
           \]

        \end{frame}

        \begin{frame}
            \frametitle{Boundary Values}

            Obtained from central density and hydrostatic equation

            % look up how to itemize & put in a box
            \[
                \begin{array}{l l}
                    \xi &= 0 \\
                    \theta &= 1 \\
                    \frac{d\theta}{d\xi} &= 0
                \end{array}
            \]

        \end{frame}

        \begin{frame}
            \frametitle{Runge-Kutta Solution}

        \end{frame}

        \begin{frame}
           \frametitle{Problem!}

            Singularity at \(\xi_0=0\):

            \[ \phi'=-\left(\frac{2}{\xi_0}\phi\right)-\theta_0^n\]

            Need to work around this somehow:
            \pause
            \begin{itemize}
                \item Taylor expand at \(\xi=0\) and take limit as \(\xi\rightarrow 0\): \(\phi'\rightarrow -\frac{1}{3}\)
                \pause
                \item Offset the starting point: \(0< \xi_0 \ll 1\)
            \end{itemize}

        \end{frame}

    \section{Results}

        \begin{frame}
            \frametitle{Placeholder}

        \end{frame}

    \section{Discussion}

        \begin{frame}
            \frametitle{Questions?}

        \end{frame}

\end{document}
